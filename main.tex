\documentclass{article}
\usepackage[utf8]{inputenc}
\usepackage[spanish]{babel}
\usepackage{listings}
\usepackage{graphicx}
\graphicspath{ {images/} }
\usepackage{cite}

\begin{document}

\begin{titlepage}
    \begin{center}
        \vspace*{1cm}
            
        \Huge
        \textbf{Calistenia}
            
        \vspace{0.5cm}
        \LARGE
        Pasar del estado inicial, al estado final, solo con instrucciones
            
        \vspace{1.5cm}
            
        \textbf{Jonathan Enrique Macias Diaz}
            
        \vfill
            
        \vspace{0.8cm}
            
        \Large
        Despartamento de Ingeniería Electrónica y Telecomunicaciones\\
        Universidad de Antioquia\\
        Medellín\\
        Marzo de 2021
            
    \end{center}
\end{titlepage}

\tableofcontents
\newpage
\section{Introducción}\label{intro}
En este proceso se busca guiar a una persona como realizar una actividad, por medio de instrucciones, como si fuese una máquina, a la persona no se le explicara de que se basa la actividad ni se le explicara que debe interpretar cada instrucción

\section{Pasos para llevar  los objetos de la posición A a la posición B} \label{contenido}
1.	Conseguir 2 tarjetas, de igual tamaño, peso, si es posible de igual grosor y en perfecto estado, conseguir una hoja blanca, lisa, sin deformidades, del tamaño 10x10 o mayor\newline
2.	Colocar las 2 tarjetas, una encima de la otra y perfectamente alineada, encima de una mesa o superficie plana horizontal que sea estable, sin irregularidades en su superficie, y ubicar la hoja encima de las tarjetas, con la misma orientación que las tarjetas, tratando que las tarjetas queden lo más alejado de los bordes de la hoja.\newline
*A partir de aquí, solo con la mano derecha\newline
3.	Con el dedo índice y medio de la mano derecha, presionaremos la hoja blanca por uno de sus extremos, presionaremos y moveremos hasta que queden descubiertas las tarjetas y sin que se caiga de la mesa.\newline
4.	Despegamos los dedos de la hoja, y procedemos a tomar las tarjetas con el dedo gordo por un horizontal, el dedo medio junto al anular por el lado contrario al dedo gordo y el dedo índice por la parte superior o inferior de la tarjeta, las levantamos y la ubicamos de manera vertical, retiramos el dedo anular y hacemos presión hacía dentro con el dedo gordo y medio, y con el índice hacemos presión hacía abajo, para alinearlas. \newline
5.	Ahora, levantamos las tarjetas y de manera vertical, las ubicamos encima de la hoja, de manera que la distancia hacía los bordes de la hoja sean casi iguales \newline
6.	Con el dedo índice le haremos un poco de presión a la tarjeta más alejada de la palma de tu mano, hacía la hoja, mientras que con el dedo gordo y medio, nos dirigimos hacía la parte baja de la tarjeta que no le estamos haciendo presión, y ejerciendo un poco de presión la alejaremos de la parte inferior de la tarjeta cuál estamos presionando, al menos hasta que quepa el dedo gordo o medio \newline
7.	Seguidamente, a la tarjeta que le estamos haciendo presión con el dedo índice, la inclinamos hacía la tarjeta que está más cerca de la palma y retiraremos el dedo índice, para que, solo con el dedo gordo y suplantando al dedo angular, por el dedo índice para mejor agarre, ubicándolo también en la parte inferior de la tarjeta, alejemos o acerquemos la parte inferior de la misma hasta lograr el equilibrio formando una pirámide, se necesita paciencia




\bibliographystyle{IEEEtran}


\end{document}
